%%%%%%%%%%%%%%%%%%%%%%%%%%%%%%%%%%%%%%%%%%%%%%%%%%%%%%%%%%%%
% Document settings
\documentclass{ACGSeminar}

%%%%%%%%%%%%%%%%%%%%%%%%%%%%%%%%%%%%%%%%%%%%%%%%%%%%%%%%%%%%
% Own Packages

%%%%%%%%%%%%%%%%%%%%%%%%%%%%%%%%%%%%%%%%%%%%%%%%%%%%%%%%%%%%
% Own Definitions
\newcommand{\comment}[1]{}


%%%%%%%%%%%%%%%%%%%%%%%%%%%%%%%%%%%%%%%%%%%%%%%%%%%%%%%%%%%%
% BibTex
\bibliography{references}

%%%%%%%%%%%%%%%%%%%%%%%%%%%%%
% Hyphenations here
%%%%%%%%%%%%%%%%%%%%%%%%%%%%%
\hyphenation{Sa-tan-arch-aeo-li-deal-co-hell-ish}


%%%%%%%%%%%%%%%%%%%%%%%%%%%%%
% Title, Author, etc.

\begin{document}

\title{Re: Deep G-Buffers for Stable Global Illumination Approximation}

\author{Ferit Tohidi Far}

\maketitle

%%%%%%%%%%%%%%%%%%%%%%%%%%%%%%%%%%%%%%%%%%%%%%%%%%%%%%%%%%%%
% Abstract

\begin{abstract}%
G-Buffers can be used to efficiently render images with a large amount of light sources compared to other local illumination methods. This is possible thanks to a 
process called "deferred rendering". By using Deep G-Buffers we can even approximate global illumination more efficiently than traditional methods like pathtracing. 
\end{abstract}

\keywords{g-buffer, deep g-buffer, pathtracing, global illumination, shading}
\tableofcontents

\newpage

%%%%%%%%%%%%%%%%%%%%%%%%%%%%%%%%%%%%%%%%%%%%%%%%%%%%%%%%%%%%
% Introduction
\label{cha:introduction}
\section{Global Illumination}
	Global Illumination is achieved by taking into account every light, as well as every reflection of every light. To do this there are different methods with the
	most common ones being pathtracing and photonmapping.
	\subsection{Pathtracing}
		
	\subsection{Photonmapping}
\section{Deferred Rendering}
	\subsection{How deffered rendering handles lighting more efficiently}

\section{G-Buffer}
	\subsection{Frame-Buffer}
	\subsection{Z-Buffer}
	\subsection{Position-Buffer}
	\subsection{Normal-Buffer}
	\subsection{Diffuse-buffer}
	\subsection{Computing local illumination using G-Buffers}
	\subsection{Computing global illumination using G-Buffers}
	\subsection{Performance comparison: G-buffers vs Pathtracing}
	\subsection{Output comparison: G-Buffers vs Pathtracing}

\section{Deep G-Buffer}
	\subsection{Concept}
	\subsection{How Deep G-Buffers improve performance}
	\subsection{Performance comparison: G-buffers vs Deep G-Buffers vs Pathtracing}
	\subsection{Output comparison: G-Buffers vs Deep G-Buffers vs Pathtracing}

%\section{My Fancy Second Section}
%My second section is even better than the first one\footnote{Proof omitted.} because it contains a %footnote as well as some citations~\cite{Squarepants03,Knuth97}.%
%Everything else is explained in Figure~\ref{fig:dummy}. Always m%ake sure that all figures are refer%enced.
%
%\begin{figure}[htb!]
%  \begin{centering}
%    \includegraphics[width=6cm]{img/thing.jpg}\par
%  \end{centering}
%  \caption{A simple dummy picture to demonstrate how to include graphics into your report.}
%  \label{fig:dummy}
%\end{figure}

%%%%%%%%%%%%%%%%%%%%%%%%%%%%%%%%%%%%%%%%%%%%%%%%%%%%%%%%%%%%
% Bibliography
\label{cha:references}
\printbibliography

\end{document}
