%%%%%%%%%%%%%%%%%%%%%%%%%%%%%%%%%%%%%%%%%%%%%%%%%%%%%%%%%%%%
% Document settings
\documentclass{ACGSeminar}

%%%%%%%%%%%%%%%%%%%%%%%%%%%%%%%%%%%%%%%%%%%%%%%%%%%%%%%%%%%%
% Own Packages

%%%%%%%%%%%%%%%%%%%%%%%%%%%%%%%%%%%%%%%%%%%%%%%%%%%%%%%%%%%%
% Own Definitions
\newcommand{\comment}[1]{}


%%%%%%%%%%%%%%%%%%%%%%%%%%%%%%%%%%%%%%%%%%%%%%%%%%%%%%%%%%%%
% BibTex
\bibliography{references}

%%%%%%%%%%%%%%%%%%%%%%%%%%%%%
% Hyphenations here
%%%%%%%%%%%%%%%%%%%%%%%%%%%%%
\hyphenation{Sa-tan-arch-aeo-li-deal-co-hell-ish}


%%%%%%%%%%%%%%%%%%%%%%%%%%%%%
% Title, Author, etc.

\begin{document}

\title{Re: Deep G-Buffers for Stable Global Illumination Approximation}

\author{Ferit Tohidi Far}

\maketitle

%%%%%%%%%%%%%%%%%%%%%%%%%%%%%%%%%%%%%%%%%%%%%%%%%%%%%%%%%%%%
% Abstract

\begin{abstract}%
G-buffers can be used to efficiently render images with large amounts of light sources. This is possible thanks to a process called "deferred rendering". Using 
g-buffers, we are only able to compute local illumination, though. By using deep g-buffers instead we can approximate global illumination, which is way more 
efficient than traditional global illumination methods like pathtracing, while of course not being physically accurate anymore. 
\end{abstract}

\keywords{g-buffer, deep g-buffer, pathtracing, global illumination, deferred shading, deferred rendering}
\tableofcontents

\newpage

%%%%%%%%%%%%%%%%%%%%%%%%%%%%%%%%%%%%%%%%%%%%%%%%%%%%%%%%%%%%
% Introduction
\label{cha:introduction}
\section{Global Illumination}
	Global Illumination is a lighting effect that is achieved by not only computing direct light, but also indirect light, meaning that it is neccesary to take
	into account how light reflects and carries information (in the most basic case: color).
	\subsection{Physically correct methods}
	In order to generate physically correct images, which is a requirement for creating photorealistic images, we need to solve the rendering equation.
	% TODO insert rendering equation here
	The most popular method for achieving this is pathtracing \cite{P2PATH}.
	\subsubsection{Pathtracing}
	\subsection{Computational difficulties of physically correct methods}
	Since we have to take into account every ray of light and its reflections, the computational difficulty becomes apparent \cite{DST}.
	% TODO insert time-complexity of pathtracing and time-complexity of deferred-rendering for comparison

\section{Deferred Rendering}
	\subsection{How deferred rendering handles lighting more efficiently}
		The goal of deferred rendering is to postpone the shading stage. %TODO insert information about graphics pipeline of gpu
		Instead of shading right away, we compute necessary geometry buffers (g-buffers) and cache 
		them for later use. For all practical purposes, g-buffers have to at least consist of a frame-buffer, a normal-buffer and a z-buffer. Using only these g-buffers, 
		we can now compute the shading separately.
	\subsection{Deferred Shading}

\section{Geometry-buffer (g-buffer)}
	Each geometry-buffer stores information of some sort for each individual pixel, meaning that they are all two-dimensional arrays.
	\subsection{Frame-Buffer}
		The frame-buffer stores color values. 
	\subsection{Z-Buffer}
		The z-buffer stores depth values.
	\subsection{Normal-Buffer}
		The normal-buffer stores surface-normals.
		Note that there are more possible buffers to choose from, but the three that were mentioned are the most essential in every g-buffer.
	\subsection{Computing local illumination using g-buffers}
		After collecting all g-buffers we can now work on illumination. To do this we need to define some light sources. We distinguish between three types of lights:
		Point-lights, spot-lights and directional-lights \cite{DST}.

\section{Deep G-Buffer}
	\subsection{Concept}
	Store multiple g-buffers \cite{NDGB}.
	\subsection{How Deep G-Buffers improve performance}

\section{Performance and Output Comparison}
	\subsection{Deep G-Buffers vs Pathtracing}

%\section{My Fancy Second Section}
%My second section is even better than the first one\footnote{Proof omitted.} because it contains a %footnote as well as some citations~\cite{Squarepants03,Knuth97}.%
%Everything else is explained in Figure~\ref{fig:dummy}. Always m%ake sure that all figures are refer%enced.
%
%\begin{figure}[htb!]
%  \begin{centering}
%    \includegraphics[width=6cm]{img/thing.jpg}\par
%  \end{centering}
%  \caption{A simple dummy picture to demonstrate how to include graphics into your report.}
%  \label{fig:dummy}
%\end{figure}

%%%%%%%%%%%%%%%%%%%%%%%%%%%%%%%%%%%%%%%%%%%%%%%%%%%%%%%%%%%%
% Bibliography
\label{cha:references}
\printbibliography

\end{document}
